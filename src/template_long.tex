% Template_long.tex
% Template for long documents
%
% Created 1. September 2020

\documentclass[a4paper]{book} % document class
\usepackage{template_package} % all necessary things

% Large document with title page
% and table of contents
\largedocdesign
%\onepagedesign

% *** DETAILS *** %
\version{1.0.0}
\departement{ETH Zürich D-ITET}
\university{Departement Informationstechnologie und Elektrotechnik D-ITET}
\doctype{Semesterarbeit}
\title{Titel des Dokuments}
\subtitle{Der Untertitel befindet sich unterhalb des Titels}
\author{Stefan Gloor}
\semester{Herbstsemester 2020}
% *************** %

% extract ID and branch name from .git directory, 
% located above working directory
\gitinfo

% Create headers and footers
\createtitlepagestyle
\createpagestyle

\begin{document} 
    \thispagestyle{TitlePageStyle} % first page has title page style
    \maketitle % draw title and subtitle
    \tocloftpagestyle{fancy} % Page with TOC has same style as title page
    \renewcommand{\cfttoctitlefont}{\sffamily \bfseries \Large}
    \renewcommand{\cftchapfont}{\sffamily \bfseries \large}
    \renewcommand{\cftsecfont}{\sffamily \bfseries}
    \renewcommand{\cftsubsecfont}{\sffamily}
    \renewcommand{\cftsubsubsecfont}{\sffamily}
    \renewcommand{\cftchappagefont}{\sffamily \bfseries \large}
    \renewcommand{\cftsecpagefont}{\sffamily \bfseries}
    \renewcommand{\cftsubsecpagefont}{\sffamily}
    \renewcommand{\cftsubsubsecpagefont}{\sffamily}
    \renewcommand{\cftchapaftersnum}{\bfseries\large.}
    \renewcommand{\cftsecaftersnum}{\bfseries.}
    \renewcommand{\cftsubsecaftersnum}{.}
    \renewcommand{\cftsubsubsecaftersnum}{.}
    \renewcommand{\cftbeforechapskip}{4mm}
    \renewcommand{\cftbeforesecskip}{2mm}
    \renewcommand{\cftbeforesubsecskip}{2mm}
    \renewcommand{\cftbeforesubsubsecskip}{2mm}
    \setlength{\cftbeforetoctitleskip}{0pt}
    \setlength{\cftaftertoctitleskip}{10pt}
    \tableofcontents
    \newpage
    

    % *** CONTENT *** %

    \chapter{Anfänge}
        \section{Einleitung}
            \subsection{Beispiel}
                \blindtext
            \subsection{Vorschlag}
                \blindtext
                \subsubsection{Hinweis}
                    \blindtext
            \subsection{Überlegungen}
                \blindmathpaper
        \section{Hauptteil}
            \blindtext
        \section{Fazit}
    \chapter{Nächstes Kapitel}
        \section{Erste Schritte}
            \blindmathpaper
        \section{Weitere Schritte}
            \blindtext

\end {document}